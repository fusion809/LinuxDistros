\documentclass[12pt,a4paper,landscape]{article}
\setcounter{secnumdepth}{0}
\usepackage{gensymb}
\usepackage{pdflscape}
\usepackage{amsmath}
\usepackage{amssymb}
\usepackage{enumitem}
\usepackage{graphicx}
\usepackage{sansmath}
\usepackage{pst-eucl}
\usepackage{multicol}
\usepackage[UKenglish]{isodate}
\usepackage[UKenglish]{babel}
\usepackage{float}
\usepackage[T1]{fontenc}
\usepackage{setspace}
\usepackage{sectsty}
\usepackage{longtable}
\usepackage[colorlinks=true,linkcolor=blue,urlcolor=black,bookmarksopen=true]{hyperref}
\sectionfont{%			            % Change font of \section command
	\usefont{OT1}{phv}{b}{n}%		% bch-b-n: CharterBT-Bold font
	\sectionrule{0pt}{0pt}{-5pt}{3pt}}
\subsectionfont{
	\usefont{OT1}{phv}{b}{n}}
\newcommand{\MyName}[1]{ % Name
	\usefont{OT1}{phv}{b}{n} \begin{center}of {\LARGE  #1}\end{center}
	\par \normalsize \normalfont}
\newcommand{\MyTitle}[1]{ % Name
	\Huge \usefont{OT1}{phv}{b}{n} \begin{center}#1\end{center}
	\par \normalsize \normalfont}
\newcommand{\NewPart}[1]{\section*{\uppercase{#1}}}
\newcommand{\NewSubPart}[1]{\subsection*{\hspace{0.2cm}#1}}
\renewcommand{\baselinestretch}{2.0}
\usepackage[margin=0.1in]{geometry}
\title{Independent Linux distros}
\date{}

\begin{document}
	\maketitle
	  
			In the following table, PMS refers to package management system and rec. is recommended. When an init system is listed as recommended, the choice is ultimately up to the user.
	\begin{longtable}{|p{3.5cm}|p{2.3cm}|p{1.6cm}|p{2.4cm}|p{2.0cm}|p{14.7cm}|}
			\hline
			\textbf{Distro} & \textbf{PMS} & \textbf{Release} & \textbf{Init system} & \textbf{Founded} & \textbf{Other characteristics}\\\hline
			4MLinux & TXZ\footnote{Which I use to indicate that tar.xz or .txz files are used for packaging} & Fixed & Busybox & 2010, PL & Focuses on the 4 Ms: multimedia, maintenance, miniserver and mystery (small games). Lightweight. \\\hline 
			Alpine Linux & APK & Fixed & Busybox, OpenRC & 2005, NO & Uses musl as C system library. Designed to be small, simple and secure and used for servers, routers, embedded devices, etc. \\\hline
			ALT Linux & RPM/APT & Fixed & systemd & 2001, RU & Early fork of Mandrake. \\\hline 
			Arch Linux & Pacman & Rolling & systemd & 2002, CA & Follows KISS principle. \\\hline
			Batocera.linux & ? & Fixed & SysV & 2016, FR & Minimalist and dedicated to retrogaming. Can run on desktops, laptops, Raspberry Pi, etc. \\\hline
			Bedrock Linux & NA & Fixed & NA & 2009, US & Meta distribution which allows one to use features from other different distributions. \\\hline
			Chimera Linux & APK (bin); cports (src) & Rolling & Dinit & 2021, ES & Uses musl C library and FreeBSD userland. \\\hline
			Clear Linux & swupd & Rolling & systemd & 2015, US & Minimalist distribution designed with performance and cloud use in mind. \\\hline
			CRUX & TXZ & Fixed & SysV & 2002, SE & Lightweight distro targeted towards advanced users. It uses tar.xz packaging system and has a relatively small collection of packages (making it similar to Slackware).\\\hline
			Debian & dpkg/APT & Fixed & systemd & 1993, US & Basis of most Linux distros, such as Ubuntu.\\\hline
			Dragora GNU/Linux-Libre & TLZ & Fixed & SysV & 2009, AR & Uses only FOSS. \\\hline
			EasyOS & PET & Fixed & SysV & 2018, AU & Experimental, uses Puppy tech. Containers can be used to run apps or desktops. \\\hline
			Exherbo & Paludis & Rolling & systemd rec\footnote{As a source distribution that is aimed for configurability, the user gets to choose the init system. Although, systemd seems to be recommend} & 2009, DK & Essentially aiming to be like Gentoo, but with a better package manager. \\\hline
			Fedora & RPM/dnf & Fixed & systemd & 2003, US & Based on earlier Red Hat Linux. Basis of RHEL. \\\hline
			Gentoo & Portage & Rolling & OpenRC rec\footnote{Source distro that lets users to choose init system. System is installed from stage3 tarballs and both OpenRC and systemd stage3 tarballs are provided.} & 2002, US & Most popular source-based distribution. \\\hline
			Guix System & Guix & Fixed & Shepherd & 2015, FR & Whole system configured in Guile language; reproducible builds. \\\hline
			Hyperbola GNU/Linux-libre & Pacman & Fixed & OpenRC & 2017, BR & Only uses FOSS. Previously used patches from Debian and snapshots from Arch. \\\hline
			IPFire & pakfire & Fixed & Other & <2015, DE & Uses WebUI. \\\hline
			KaOS & Pacman & Rolling & systemd & 2003, US & Uses KDE as default desktop. \\\hline 
			LibreELEC & None & Fixed & systemd & 2016, US & Designed for running Kodi.\\\hline
			LinuxConsole & opkg & Fixed & ? & 2004, FR & Default interface is MATE, and it is a live CD distribution. \\\hline
			Linux From Scratch & None & Fixed & Up to user & 1999, CA & Self-made Linux system. \\\hline
			Mageia & RPM/dnf & Fixed & systemd & 2011, FR & urpmi was original package manager; forked from Mandriva Linux by former employees of its maintainer. \\\hline
			NixOS & Nix & Fixed & systemd & 2003, NL & Whole system configured in Nix language; reproducible builds. \\\hline
			openmamba GNU/Linux & RPM/dnf & Rolling & systemd & <2008, IT & Offers KDE and LXQt desktops. \\\hline
			OpenMandriva Lx & RPM/dnf & Fixed \& rolling. & systemd & 2013, FR & Smaller development team than Mageia. Built with Clang instead of GCC. \\\hline
			openSUSE & RPM/zypper & Fixed \& rolling. & systemd & 1994, DE & Has openSUSE build service for building custom packages. Btrfs is default filesystem on it. \\\hline
			OviOS Linux & Pacman & Fixed & SysV & <2017, CA & Independent storage OS aimed at performance. \\\hline
			paldo GNU/Linux & Upkg & Rolling & ? & 2004, CH & Hybrid (binary/source) approach to package management. Used for data rescue and uses GNOME. \\\hline
			PCLinuxOS & RPM/APT & Rolling & SysV & 2003, US & Designed to be beginner-friendly. \\\hline
			Peropesis & None & Fixed & SysV & 2021, LT & Minimalist, live medium and command line-based. \\\hline
			Photon OS & rpm-ostress & Fixed & systemd & <2016, US & Minimal and optimized for VMware platforms. \\\hline
			Pisi Linux & PiSi & Fixed & SysV & 2013?, TR & Based on defunct Pardus Linux (that used PiSi). \\\hline
			PLD Linux & RPM/Poldek & Rolling & Other\footnote{Or so says DistroWatch, cannot find information on this from other sources.} & 1998, PL & Aimed at advanced users, designed to be configurable. \\\hline
			Plop Linux & NA\footnote{Booted as live medium, so none would probably be necessary. Also worth noting that DistroWatch does not specify a package management system for it.} & Fixed & SysV & <2009, AT& Small distro for live use, so that people can rescue data, repair damaged systems, automate tasks, etc. \\\hline
			Puppy Linux & PET & Fixed & SysV & 2003, AU & Lightweight, designed to be run from live sessions and can be run from RAM. \\\hline
			recalbox & ?\footnote{Field left blank on DistroWatch} & Fixed & SysV & <2019, global\footnote{According to DistroWatch} & Dedicated to running video games on emulated retro/console platforms and running Kodi. \\\hline
			rlxos & Flatpak & Fixed & systemd & 2023?, IN & Uses immutable filesystem, Btrfs by default and Bolt AI assistant. \\\hline
			ROSA & RPM/dnf & Fixed & systemd & 2010, RU & Previously used urpmi. Developed by a Russian company. \\\hline
			Slackware Linux & TXZ & Fixed & SysV & 1993, US & Oldest continually developed Linux distro. \\\hline
			SliTaz GNU/Linux & TazPKG & Fixed & Busybox? & 2008, CH & Lightweight, loads from RAM with just 256MB required. \\\hline  
			Solus & eopkg & Rolling & systemd & 2015, IE & Started the development of the Budgie desktop environment. \\\hline
			Tiny Core Linux & TCE & Fixed & Busybox & 2009, US & Uses 16MB RAM to boot and is designed to be the lightest distro possible. \\\hline
			Void & XBPS & Rolling & runit & 2008, ES & Offers editions using glibc and musl. \\\hline
			Venom Linux & scratchpkg & Rolling & SysV & 2021, MY & Uses Openbox for GUI and has textual installer. \\\hline
			Vine Linux & RPM/APT & Rolling & ? & 1998, JP & Had fixed releases, with the last released in 2017. \\\hline
	\end{longtable}
	\end{document}
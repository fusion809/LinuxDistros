\documentclass[12pt,a4paper]{article}
\setcounter{secnumdepth}{0}
\usepackage{gensymb}
\usepackage{pdflscape}
\usepackage{amsmath}
\usepackage{amssymb}
\usepackage{enumitem}
\usepackage{graphicx}
\usepackage{sansmath}
\usepackage{pst-eucl}
\usepackage{multicol}
\usepackage[UKenglish]{isodate}
\usepackage[UKenglish]{babel}
\usepackage{float}
\usepackage[T1]{fontenc}
\usepackage{setspace}
\usepackage{sectsty}
\usepackage[colorlinks=true,linkcolor=blue,urlcolor=black,bookmarksopen=true]{hyperref}
\sectionfont{%			            % Change font of \section command
	\usefont{OT1}{phv}{b}{n}%		% bch-b-n: CharterBT-Bold font
	\sectionrule{0pt}{0pt}{-5pt}{3pt}}
\subsectionfont{
	\usefont{OT1}{phv}{b}{n}}
\newcommand{\MyName}[1]{ % Name
	\usefont{OT1}{phv}{b}{n} \begin{center}of {\LARGE  #1}\end{center}
	\par \normalsize \normalfont}
\newcommand{\MyTitle}[1]{ % Name
	\Huge \usefont{OT1}{phv}{b}{n} \begin{center}#1\end{center}
	\par \normalsize \normalfont}
\newcommand{\NewPart}[1]{\section*{\uppercase{#1}}}
\newcommand{\NewSubPart}[1]{\subsection*{\hspace{0.2cm}#1}}
\renewcommand{\baselinestretch}{2.0}
\usepackage[margin=0.1in]{geometry}
\title{Independent Linux distros}
\date{}

\begin{document}
	\maketitle
	
	In the following table, PMS refers to package management system and rec. is recommended. When an init system is listed as recommended, the choice is ultimately up to the user.  
	\begin{landscape}
	\begin{table}
		\begin{tabular}{|p{3.5cm}|p{2.3cm}|p{1.6cm}|p{2.3cm}|p{1.9cm}|p{14.8cm}|}
			\hline
			Distro & PMS & Release & Init system & Founded & Other characteristics\\\hline
			Alpine Linux & APK & Fixed & Busybox, OpenRC & 2005, NO & Uses musl as C system library. Designed to be small, simple and secure and used for servers, routers, embedded devices, etc. \\\hline
			Arch Linux & pacman & Rolling & systemd & 2002, CA & Follows KISS principle. \\\hline
			Chimera Linux & APK (bin); cports (src) & Rolling & Dinit & 2021, ES & Uses musl C library and FreeBSD userland. \\\hline
			Debian & dpkg/APT & Fixed & systemd & 1993, USA & Basis of most Linux distros, such as Ubuntu.\\\hline
			Dragora GNU/Linux-Libre & TLZ & Fixed & SysV & 2009, AR & Uses only FOSS. \\\hline
			EasyOS & PET & Fixed & SysV & 2018, AU & Experimental, uses Puppy tech. Containers can be used to run apps or desktops. \\\hline
			Exherbo & Paludis & Rolling & systemd rec. & 2009, DK & Essentially aiming to be like Gentoo, but with a better package manager. \\\hline
			Fedora & RPM/dnf & Fixed & systemd & 2003, USA & Based on earlier Red Hat Linux. Basis of RHEL. \\\hline
			Gentoo & Portage & Rolling & OpenRC rec. & 2002, USA & Most popular source-based distribution. \\\hline
			Guix System & Guix & Fixed & Shepherd & 2015, FR & Whole system configured in Guile language; reproducible builds. \\\hline
			Linux From Scratch & None & Fixed & Up to user & 1999, CA & Self-made Linux system. \\\hline
			Mageia & RPM/dnf & Fixed & systemd & 2011, FR & urpmi was original package manager; forked from Mandriva Linux by former employees of its maintainer. \\\hline
			NixOS & Nix & Fixed & systemd & 2003, NL & Whole system configured in Nix language; reproducible builds. \\\hline
			OpenMandriva Lx & RPM/dnf & Fixed \& rolling. & systemd & 2013, FR & Smaller development team than Mageia. Built with Clang instead of GCC. \\\hline
			openSUSE & RPM/zypper & Fixed \& rolling. & systemd & 1994, DE & Has openSUSE build service for building custom packages. \\\hline
			paldo GNU/Linux & Upkg & Rolling & ? & 2004, CH & Hybrid (binary/source) approach to package management. Used for data rescue and uses GNOME. \\\hline
			PCLinuxOS & RPM/APT & Rolling & SysV & 2003, USA & Designed to be beginner-friendly. \\\hline
			Puppy Linux & PET & Fixed & SysV & 2003, AU & Lightweight, designed to be run from live sessions and can be run from RAM. \\\hline
			Slackware Linux & TXZ & Fixed & SysV & 1993, USA & Oldest continually developed Linux distro. \\\hline
			Solus & eopkg & Rolling & systemd & 2015, IE & Started the development of the Budgie desktop environment. \\\hline
			Void & XBPS & Rolling & runit & 2008, ES & Offers editions using glibc and musl. \\\hline
			Venom Linux & scratchpkg & Rolling & SysV & 2021, MY & Uses Openbox for GUI and has textual installer. \\\hline
			Vine Linux & RPM/APT & Rolling & ? & 1998, JP & Had fixed releases, with the last released in 2017. \\\hline
		\end{tabular}
	\end{table}
	\end{landscape}
	\end{document}